\textbf{Problem 8}

%\begin{itemize}
% \item[\textbf{a}.]
% \item[\textbf{b}.] 
%\end{itemize}

\begin{solution}\ 
\begin{itemize}
 \item[\textbf{a}.] We want to find $p_Z(z) := P(X + Y = z)$ and have that $R_X = R_Y = \{1,2,3,4\}$. The sum of the distributions for $p_X(x) = \sum_i P(X = i)$ and $p_Y(x) = \sum_i P(Y = i)$ forms the distribution for $z$, equal to (well, at least represented one way because there are other ways of representing $p_Z(z)$):
 \[p_Z(z) = \sum_{i\in R_X} P(X = i \text{ and } Y = z - i)\]
 since $X$ and $Y$ are independent this equals
 \[p_Z(z) = \sum_{i\in R_X} P(X = i)P(Y = z - i)\]
 this is equivalent to:
 \[p_Z(z) = \sum_{i\in R_X} p_X(i)p_Y(z-i)\]
 We have that $R_Z = \{2,3,4,5,6,7,8\}$ (since the ranges of $X$ and $Y$, which are summed, must also be summed) and still have that $R_X=\{1,2,3,4\}$. Evaluating, we get:
\[p_Z(2) = \sum_{i\in R_X} p_X(i)p_Y(2-i) = \bigg(\frac{1}{4}\cdot\frac{1}{4}\bigg) + \bigg(\frac{1}{4}\cdot0\bigg) + \bigg(\frac{1}{4}\cdot0\bigg) + \bigg(\frac{1}{4}\cdot0\bigg)=\frac{1}{16}\]

\[p_Z(3) = \sum_{i\in R_X} p_X(i)p_Y(3-i) = \bigg(\frac{1}{4}\cdot\frac{1}{4}\bigg) + \bigg(\frac{1}{4}\cdot\frac{1}{4}\bigg) + \bigg(\frac{1}{4}\cdot0\bigg) + \bigg(\frac{1}{4}\cdot0\bigg)=\frac{1}{16}+\frac{1}{16}=\frac{1}{8}\]

\[p_Z(4) = \sum_{i\in R_X} p_X(i)p_Y(4-i) = \bigg(\frac{1}{4}\cdot\frac{1}{4}\bigg) + \bigg(\frac{1}{4}\cdot\frac{1}{4}\bigg) + \bigg(\frac{1}{4}\cdot\frac{1}{4}\bigg) + \bigg(\frac{1}{4}\cdot0\bigg)=\frac{1}{16}+\frac{1}{16}+\frac{1}{16}=\frac{3}{16}\] 

\[p_Z(5) = \sum_{i\in R_X} p_X(i)p_Y(5-i) = \bigg(\frac{1}{4}\cdot\frac{1}{4}\bigg) + \bigg(\frac{1}{4}\cdot\frac{1}{4}\bigg) + \bigg(\frac{1}{4}\cdot\frac{1}{4}\bigg) + \bigg(\frac{1}{4}\cdot\frac{1}{4}\bigg)=\frac{1}{16}+\frac{1}{16}+\frac{1}{16}+\frac{1}{16}=\frac{1}{4}\]

\[p_Z(6) = \sum_{i\in R_X} p_X(i)p_Y(6-i) = \bigg(\frac{1}{4}\cdot0\bigg) + \bigg(\frac{1}{4}\cdot\frac{1}{4}\bigg) + \bigg(\frac{1}{4}\cdot\frac{1}{4}\bigg) + \bigg(\frac{1}{4}\cdot\frac{1}{4}\bigg)=\frac{1}{16}+\frac{1}{16}+\frac{1}{16}=\frac{3}{16}\]

\[p_Z(7) = \sum_{i\in R_X} p_X(i)p_Y(7-i) = \bigg(\frac{1}{4}\cdot0\bigg) + \bigg(\frac{1}{4}\cdot0\bigg) + \bigg(\frac{1}{4}\cdot\frac{1}{4}\bigg) + \bigg(\frac{1}{4}\cdot\frac{1}{4}\bigg)=\frac{1}{16}+\frac{1}{16}=\frac{1}{8}\]

\[p_Z(8) = \sum_{i\in R_X} p_X(i)p_Y(8-i) = \bigg(\frac{1}{4}\cdot0\bigg) + \bigg(\frac{1}{4}\cdot0\bigg) + \bigg(\frac{1}{4}\cdot0\bigg) + \bigg(\frac{1}{4}\cdot\frac{1}{4}\bigg)=\frac{1}{16}\]

We have that:

\[p_Z(z) = \begin{cases}\frac{1}{16}, & \text{if $z$ = 2 or $z$ = 8}\\\frac{1}{8}, & \text{if $z$ = 3 or $z$ = 7}\\\frac{3}{16}, & \text{if $z$ = 4 or $z$ = 6}\\\frac{1}{4}, & \text{if $z$ = 5}\end{cases}\]
 
% \[p_z(z) = \sum_i P(X = z -i \text{ and } Y = i)\]
 %\[p_Z(z) = \sum_x p_X(x)p_{Y|X}(z-x|x)\]
 %and that $R_Y = R_X = \{1,2,3,4\}$ and want to find the formula for %$p_Z(z) := P(X + Y =z)$.
%\[p_Z(1) = \sum_1 p_X(1)p_{Y|X}(z-1|1) \]
 
 
 \item[\textbf{b}.] Here are the results for 10000 iterations of programmatical experimentation using $R_X = R_Y = \{1,2,3,4\}$ and that $P(X + Y = z)$.\\  
Value: 2\\
Number: 675\\
Percent Total: 6.75\%\\

Value: 3\\
Number: 1252\\
Percent Total: 12.52\%\\

Value: 4\\
Number: 1945\\
Percent Total: 19.45\%\\

Value: 5\\
Number: 2466\\
Percent Total: 24.66\%\\

Value: 6\\
Number: 1877\\
Percent Total: 18.77\%\\

Value: 7\\
Number: 1191\\
Percent Total: 11.91\%\\

Value: 8\\
Number: 594\\
Percent Total: 5.94\%\\
 \textbf{See repository for code}
\end{itemize}
\end{solution}